% -*- TeX:US -*-
% TeX template to create a flow chart 
%----------------Preamble-------------------%
\documentclass{minimal}
\usepackage[paperwidth=230mm,paperheight=150mm,margin=1mm]{geometry}
\usepackage{tikz}
\usepackage{amsmath,bm,times}
\usepackage{verbatim}
\usepackage{graphicx}
\usepackage{setspace}
\usetikzlibrary{shapes,arrows,shapes.geometric}


\begin{document}
% set layers to draw the block diagrams
    \pgfdeclarelayer{background}
    \pgfdeclarelayer{foreground}
    \pgfsetlayers{background,main,foreground}

% Create required blocks for the different parts
% Building blocks
    \tikzstyle{blocklarge}  =   [draw, 
                                 minimum width=70mm,
                                 minimum height=10mm,
                                 text width=65mm,
                                 text centered
                                ]
    \tikzstyle{blockmed}    =   [draw,
                                 minimum width=30mm,
                                 minimum height=30mm,
                                 text width=25mm,
                                 text centered
                                ]
    \tikzstyle{blocksmall}  =   [draw,
                                 minimum width=15mm,
                                 minimum height=10mm,
                                 text width=10mm,
                                 text centered
                                 ]
% Distances
\def\distsmall{15mm}
\def\distmed{35mm}
\def\distlarge{75mm}
% The final block diagram
        \begin{tikzpicture}
% Earth Observations Module
            \node   (eosmodule) 
                    [blocklarge,
                     fill=blue!20
                    ] 
                    {Earth Observations Module};

% Climate Forecasting Module
            \node   (climforecast) 
                    [blocklarge,
                     fill=green!20,
                     right of=eosmodule, 
                     node distance=\distlarge] 
                    {Climate Forecasting Module};

% Hydrologic Forecasting Module
            \node   (hydforecast) 
                    [blocklarge,
                     fill=green!20,
                     below of=climforecast, 
                     node distance=\distlarge] 
                    {Climate Forecasting Module};

% Optimization Module
            

% Visualization Module         


        \end{tikzpicture}



\end{document}
