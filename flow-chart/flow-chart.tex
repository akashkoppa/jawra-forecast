% -*- TeX:US -*-
% TeX template to create a flow chart 
%----------------Preamble-------------------%
\documentclass{minimal}
\usepackage[paperwidth=230mm,paperheight=160mm,margin=1mm]{geometry}
\usepackage{tikz}
\usepackage{amsmath,bm,times}
\usepackage{verbatim}
\usepackage{graphicx}
\usepackage{setspace}
\usetikzlibrary{shapes,arrows,shapes.geometric}


\begin{document}
% set layers to draw the block diagrams
    \pgfdeclarelayer{background}
    \pgfdeclarelayer{foreground}
    \pgfsetlayers{background,main,foreground}

% Create required blocks for the different parts
% Building blocks
    \tikzstyle{blocklarge}  =   [draw, 
                                 minimum width=70mm,
                                 minimum height=5mm,
                                 text width=65mm,
                                 text centered
                                ]
    \tikzstyle{blockmed}    =   [draw,
                                 minimum width=30mm,
                                 minimum height=5mm,
                                 text width=25mm,
                                 text centered
                                ]
    \tikzstyle{blocksmall}  =   [draw,
                                 minimum width=15mm,
                                 minimum height=5mm,
                                 text width=10mm,
                                 text centered
                                 ]
% Distances
\def\distsmall{15mm}
\def\distmed{30mm}
\def\distlarge{75mm}
\def\vdistsmall{5mm}
\def\vdistlarge{65mm}
% The final block diagram
        \begin{tikzpicture}
%%----------------- Earth Observations Module
% Header
            \node   (eosmodule) 
                    [blocklarge,
                     fill=blue!20
                    ] 
                    {\textbf{Earth Observations Module}};
 % List of Precipitation (P) Datasets 
            \path   (eosmodule.south east)+(-0.5*\distmed,-4*\vdistsmall)  
                    node (pdata)
                    [blockmed,
                     fill=blue!20
                    ]
                    {\textbf{P Datasets} \\
                      CHIRPS.v2.0  \\
                      CMORPH  \\
                      PERSIANN \\
                      PERSIANN.CDR \\
                      PERSIANN.CCS \\
                      TRMM.3B42RT \\
                      TRMM.3B43
                    };

% List of Evapotranspiration (ET) Datasets 
            \path   (eosmodule.south west)+(0.5*\distmed,-4*\vdistsmall)  
                    node (etdata)
                    [blockmed,
                     fill=blue!20
                    ]
                    {\textbf{ET Datasets} \\
                      AVHRR.NTSG  \\
                      BESS \\
                      CSIRO.PML \\
                      FluxCom.RS \\
                      GLEAM \\
                      MODIS.MOD16 \\
                      SSEBOp.v4.0
                    };

% Selection of the Best P and ET Datasets using the Budyko function
            \node   (budyko)
                    [blocklarge,
                     fill=blue!20,
                     below of=eosmodule,
                     node distance=0.65*\distlarge
                    ]
                    {\textbf{Selection of P and ET Datasets based on the Budyko hypothesis} \\
                    (Koppa and Gebremichael, 2017)
                    };

% Best P and ET Datasets 
            \path   (budyko.south west)+(0.5*\distmed,-1.1*\vdistsmall)  
                    node (bestet)
                    [blockmed,
                     fill=blue!20
                    ]
                    {\textbf{Best ET Dataset} \\
                        (GLEAM) 
                    };

            \path   (budyko.south east)+(-0.5*\distmed,-1.1*\vdistsmall)  
                    node (bestp)
                    [blockmed,
                     fill=blue!20
                    ]
                    {\textbf{Best P Dataset} \\
                      TRMM.3B42RT
                    };
%%----------------- Climate Forecasting Module
            \node   (climmodule) 
                    [blocklarge,
                     fill=green!20,
                     right of=eosmodule, 
                     node distance=\distlarge] 
                    {Climate Forecasting Module};

%%----------------- Hydrologic Forecasting Module
            \node   (hydmodule) 
                    [blocklarge,
                     fill=brown!20,
                     below of=climmodule, 
                     node distance=\vdistlarge] 
                    {Hydrologic Forecasting Module};

%%----------------- Optimization Module
            \node   (optmodule)
                    [blocklarge,
                     fill=red!20,
                     right of=climmodule,
                     node distance=\distlarge
                    ]
                    {Optimization Module};

%%----------------- Visualization Module
            \node   (vismodule)
                    [blocklarge,
                     fill=blue!50,
                     below of=optmodule,
                     node distance=\vdistlarge
                    ]
                    {Visualization Module};


        \end{tikzpicture}



\end{document}
